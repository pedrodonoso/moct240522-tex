\secnumberlesssection{\textbf{Introducción}}

% Debe proporcionar a un lector los antecedentes suficientes para poder contextualizar en general la situación tratada, a través de una descripción breve del área de trabajo y del tema particular abordado, siendo bueno especificar la naturaleza y alcance del problema; así como describir el tipo de propuesta de solución que se realiza, esbozar la metodología a ser empleada e introducir a la estructura del documento mismo de la memoria.

% En el fondo, que el lector al leer la Introducción pueda tener una síntesis de cómo fue desarrollada la memoria, a diferencia del Resumen dónde se explicita más qué se hizo, no cómo se hizo.

La simulación asistida por computación es una herramienta muy utilizada en distintas áreas de la ciencia que tiene como objetivo describir de forma más precisa la realidad. En la medicina por ejemplo se utiliza la información obtenida de tomografías computarizadas o imágenes de resonancia magnética para obtener una malla de puntos o nodos que nos ayudará a representar virtualmente el objeto escaneado, a través de geometrías como hexaedros, prismas, pirámides, tetraedros o un grupo de estos, a esto se le denomina malla geométrica.
Una malla puede ser utilizada para simulación o visualización, en el primer caso los nodos internos no son utilizados, se utilizan solo mallas superficiales, es decir, mallas conformadas generalmente por triángulos o paralelogramos que representan, como dice el nombre, solo el exterior de algún objeto a modelar, la aplicación más conocida de este tipo de mallas es en la industria del entretenimiento, animación y videojuegos.
Las mallas volumétricas en su contra parte utilizan los nodos internos y elementos invisibles de la malla. La mayor aplicación es la simulación, en especial el modelamiento de fenómenos físicos para el análisis estructural.

% qué se verá en la memoria
En la presente memoria se busca integrar una etapa a posteriori de la generación de la malla Octree mixta para evitar elementos inválidos, ubicarlos y refinarlos, con el objetivo único de lograr el nivel de refinamiento requerido manteniendo la conformidad de la malla. 
 
% estrutura del documento
La estructura de esta memoria consta de 5 capítulos. En el \textit{Capítulo 1: Definición del problema}, se detalla el contexto desde donde emerge la problemática, se explica el concepto de malla geométrica y cómo se relaciona con las simulaciones computarizadas, y se menciona el trabajo de investigación existente que aborda la misma problemática y con la que se realizará la comparación. Además, se establece el objetivo general y los objetivos específicos, así como el alcance de la solución propuesta.  En el \textit{Capítulo 2: Marco conceptual}, se hace una contextualización del tipo de malla a analizar y la problemática a abordar, se revisa en términos generales el algoritmo trabajado en \cite{daines2018repairing}.
% se explicará el método de validación de la malla.
% indices de calidad
Por su parte en el \textit{Capítulo 3: Propuesta de solución} se presenta el algoritmo propuesto, así como una breve introducción y explicación de la herramienta de software utilizada para construir las mallas y refinarlas.
% También se presentarán los índices de calidad del refinamiento.
En el \textit{Capítulo 4: Validación de la solución}, se muestran y analizan las pruebas realizadas, y se realiza la comparación con el trabajo existente \cite{daines2018repairing}.
Finalmente, en el \textit{Capítulo 5: Conclusiones}, se exponen las conclusiones a las cuales se llegó en base a la comparación de los algoritmos expuestos y se proponen posibles trabajos a futuro de acuerdo a las dificultades y facilidades encontradas en el proceso de validación.